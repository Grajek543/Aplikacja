\newpage
\section{Określenie wymagań szczegółowych}		%2
%Dokładne określenie wymagań aplikacji (cel, zakres, dane wejściowe) – np. opisać przyciski, czujniki, wygląd layautu, wyświetlenie okienek. Opisać zachowanie aplikacji – co po kliknięciu, zdarzenia automatyczne. Opisać możliwość dalszego rozwoju oprogramowania. Opisać zachowania aplikacji w niepożądanych sytuacjach.


\hspace{0.60cm}Aplikacja to partpicker do pc w którym można wybrać części komputera
a aplikacja łącząca się z bazą danych sprawdzi i pokaże czy inne wybrane części są kompatybilne.

\hspace{0.60cm}Aplikacja ma wykorzystywać trzy czójniki, czójnik pozycji ekranu do automatycznego obracania,
skaner odcisku palca do szybkiego logowania i czójnik światła który będzie automatycznie dostosowywał kolor tła do ilości światła wpadającego przez ten czójnik.

\hspace{0.60cm}Aplikacja ma też automatycznie pobierać ze strony ceneo ceny komponrntów komputera
a następnie wyświetlać ją w aplikacji.

\hspace{0.60cm}Aplikacja ma pozwolić poprzez zalogowanie pozwalać na tworzenie własnych zestawów komponentów oraz 
zapisywanie ich w bazie danych, ma posiadać także koszyk w którym znajdować się będzie suma ceny wszystkich komponentów.