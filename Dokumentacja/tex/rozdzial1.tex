	\newpage
\section{Ogólne określenie wymagań}		%1
%Ogólne określenie wymagań i zakresu programu (Czyli zleceniodawca określa wymagania programu) 



\hspace{0.60cm}Projekt polega na stworzeniu listy dwukierunkowej. Listan dwókierunkowa to lista w której każdy element posiada wskaźnik poprzedniego i następnego elementu. Poniżej graficzne wytłumaczenie:
 \begin{figure}[!htb]
 	\begin{center}
 		\includegraphics[width=8cm]{rys/lista.png}
 		\caption{Lista dwókierunkowa}
 		\label{rys:rysunek001}
 	\end{center}
 \end{figure}
\hspace{0.60cm}Kolejną częścią projektu jest stworzenie konta github oraz zainstalowanie na komputerze środowiska git, a następnie sprawdzenie kilku scenariuszy związanych z nim. Owe scenariusze to: conajmniej 5 commitów, cofnięcie o 2 commity, usunięcie jednego commita, pracy na dwóch komputerach, oraz odzyskanie poprzedniej wersji projektu. Powinna być też zrobiona dokumentacja poprzez aplikację doxygen.